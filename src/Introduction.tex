\section{Introduction}

Research on the acoustics of musical instruments, a field named musical acoustics, has attempted to relate physical parameters with the characteristic sound of an instrument \cite{Caldersmith1, Christensen, Christensen3, Dickens1, Firth1, Rossing, Rossing1, Rossing3, Jannsson, Jansson:GuitarModes, Knotta, Marshall, Meyer, Meyer2, Boullosa, Richardson, Molin, Stetson, Woodhouse, J.Torres1, J.Torres, Elejabarrieta, French, Boullosa1, Boullosa2, Caldersmith, Hutchins}. For this purpose, some phenomena, mainly in the sound production and propagation, are studied based on the behavior of the instrument. The first important application of musical acoustics was done to violins and guitars \cite{Caldersmith1, Christensen, Christensen3, Dickens1, Firth1, Rossing, Rossing1, Rossing3, Jannsson, Jansson:GuitarModes, Knotta, Marshall, Meyer, Meyer2, Boullosa, Richardson, Molin, Stetson, Woodhouse, J.Torres1, J.Torres, Elejabarrieta, French, Boullosa1, Boullosa2, Caldersmith, Hutchins}, the study of these instruments provided useful information to luthiers about parameters that could modify the acoustical response of the instrument. In this sense, this work studies, using the finite element method, the normal modes of vibration for the Colombian Andean C-Bandola.

The Colombian Andean Bandola is a musical instrument that experienced a parallel development in many places and which currently presents different regionalized design adaptations. Given these conditions, it is not possible to identify a standard Colombian Andean Bandola, not even a single characteristic sound or tuning \cite{thesis:bandola}. The differences lie in the ways in which instruments are built by regional luthiers and this fact, together with reforms recently proposed by musicians and luthiers from Bogota savannah, have led to a discussion about the identity of the instrument with respect to its sonority and national musical tradition \cite{thesis:bandola}.

Normal modes are well known acoustic parameters which describe the vibrational response of the instrument. The bandola could be understood as a complex mechanical system, whose dynamic behavior depends largely on the interaction and coupling of each of its constituent elements. Knowledge about the dynamic characteristics of resonance box elements can give an idea of how structural parameters influence the behavior of the instrument as a whole. The analysis of modal coupling at the resonance box is thus proposed and developed throughout the document. 

The methodology and results were validated based on reported studies, primarily of guitars. The analysis at low resonances is emphasized through two models consisting of coupled oscillators, which could represent the vibrations of enclosed air, top and back plates at their lowest resonances. The approach of a project on this topic is intended not only contributions in academia, but also potential cultural impacts.