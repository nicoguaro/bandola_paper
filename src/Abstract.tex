\begin{abstract}
Musical acoustics have emerged as a branch that studies the sound produced for musical purposes. The specific subject of acoustics of strings instruments like guitar and violin has been extensively treated and important results of modal analysis at low resonances have shown the Fluid-structure coupling at its first two and three resonances \cite{Hutchins, Firth1, Christensen, Christensen3, Elejabarrieta} where commonly the instruments range of pitch is covered. The response at these resonances becomes important to study the sound quality of instruments; therefore, many reported models have related uncoupled modes of the air inside the acoustic box and top/back plates with the assembled instruments modes. In order to begin to understand the acoustical operation, we analyze the dynamical behavior of a ``drop shaped'' Colombian stringed instrument -the ``Bandola Andina''- at low resonances by the FEM and the coupling phenomenon is verified through the model used in \cite{Christensen, Christensen3}.

\end{abstract}

\textbf{Keyword:} Musical Acoustics, Modal Analysis, Finite Element Method, Stringed Instruments, Bandola.