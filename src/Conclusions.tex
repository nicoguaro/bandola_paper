\section{Conclusions}

We analyse modal coupling in a Colombian Andean C-Bandola at the frequency range of 0-800Hz using the Finite Element Method. First, we described the coupling for two cases using coupled systems with two and three oscillators. This allowed us to validate the numerical analysis developed later, where we verified the coupling between each first uncoupled mode of the top plate, the enclosed air and the back plate, at the lowest two and three resonances (depending on the case) of the coupled systems. The rest of the computed modes were also combinations of other uncoupled modes of each element.\\

We were able to identify the influence of the uncoupled modes on the resultant coupled modes due to the analysis based on the coupling phenomenon (described with coupled oscillators). In each case one of the uncoupled modes is dominant over the others. As a general conclusion, we remark that the coupled system of top plate-enclosed air presented a strong coupling in all the nine coupled modes. Nevertheless, in the coupled system of top plate-enclosed air-back plate the coupling could be thought to have a repulsive effect on the modes with increasing frequency.\\

Finally, we found six modes in the range of 0-800Hz for the top and back plate and three for the enclosed air. For the coupled systems: top plate-enclosed air and top plate-enclosed air-back plate, we found nine and fifteen coupled modes, respectively, in the same frequency range. Considering that modal parameters can describe the dynamic response of a structure and noting that the models of coupled systems of two and three degrees of freedom predicted the response at the lowest two and three coupled modes, we can say that a model consisting of n degrees of freedom, each one representing one of the normal modes of each uncoupled element existing at the frequency range of 0-800Hz, would describe the dynamic behavior of the bandola at the same frequency range. This idea agrees with the theory of modal analysis \cite{Ewins}.\\

In order to verify the obtained results and get more information about some characteristics of the coupled modes, for instance, the phase of vibration of the air mode and the radiation efficiency, experimental measurements are suggested as further work.
